% !TEX root =  master.tex
\newpage
\section{Konzeptuelles Modell}
% Welche Architektur und Warum?
% User-Authentifizierung?
% Welche Persistierung?
% Welche Tests und warum?

In den vorherigen Abschnitten wurden bereits einige Merkmale der Xpire-App beschrieben. Darauf aufbauend sollen im Folgenden die Zusammenhänge der verschiedenen Bestandteile näher erläutert werden und somit einen detaillierten Überblick über das konzeptionelle Modell liefern.\\
Kern der Xpire-App ist das Frontend, welches mit Hilfe von React als PWA konzipiert ist. Wie den Abbildungen \ref{fig:prot1} und \ref{fig:prot2} entnommen werden kann, besteht die Xpire-App aus 3 Ansichten: \textit{Home-Screen}, \textit{Product-Screen} und dem \textit{Create-Screen}. Diese Ansichten werden in React durch Komponenten realisiert, welche in unterschiedlichen Kontexten wiederverwendet werden können. Der Home-Screen besitzt demzufolge die Komponente AppBar und eine Komponente zur Listendarstellung der Produkte. Die Ansichten \textit{Product-Screen} und \textit{Create-Screen} werden durch dieselbe Komponente realisiert, da zum Erstellen und Anzeigen der Produktinformation ähnliche UI-Bestandteile benötigt werden. Die Darstellung passt sich hierbei automatisch anhand der übergebenen Parameter an. Hierdurch können Code-Duplikate vermieden werden und auch der Raum für Fehler wird reduziert.\\
Ein Weiterer wichtiger Bestandteil der Anwendung ist die IndexedDB, welche zur dauerhaften Persistierung der Daten verwendet wird. Wie bereits in Abschnitt \ref{chapter:datenbank} erwähnt wird hierfür das Modul \textit{Dexie} verwendet. Neben der Datenhaltung in der lokalen Datenbank wird eine aktuelle Kopie der gesamten Datenbank in einem Array im \textit{State} der App-Komponente synchron gehalten, wodurch sowohl die Listen-Komponente als auch die Product-Screen-Komponente darauf zugreifen können.

Der Nutzer soll später seine Produkte mit Hilfe des einheitlichen Barcodes hinzufügen können. Dies soll die Verwendung deutlich beschleunigen und vereinfachen, da so Informationen wie Titel, Gewicht, usw. nicht manuell eingetragen werden müssen. Um die dazu benötigten Informationen zu erhalten, wird die OpenFoodFacts-Api verwendet. Diese führt eine Datenbank mit umfassenden Produktinformationen und kann Anfragen über den Barcode entgegennehmen. Des Weiteren wird diese API verwendet, um Abbildungen der einzelnen Produkte zu erhalten. Dadurch wird dem Nutzer die Verwendung der Anwendung zusätzlich vereinfacht. Darüber hinaus ist in einer späteren Version angedacht, die Produktbilder lokal in der Datenbank zu speichern, um auch bei Offline-Betrieb die Bilder anzeigen zu können.