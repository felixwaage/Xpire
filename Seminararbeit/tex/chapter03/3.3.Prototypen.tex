% !TEX root =  master.tex
\section{Entwicklung von Prototypen}
\subsection{Figma}
Zur Erstellung der Prototypen haben wir das Design-Tool \enquote{Figma} verwendet. Dies ist ein Kollaborations-Tool für Designer und funktioniert ähnlich wie Sketch oder Adobe XD, zeichnet sich aber durch zwei signifikante Unterschiede aus\autocite[vgl.][]{?}:
\begin{itemize}[noitemsep]
	\item Es läuft zu 100 \% in Ihrem Browser (keine Installation nötig)
	\item Möglichkeit der Kollaboration mit anderen Personen in Echtzeit
\end{itemize}
Dadurch liefert Figma viele Vorteile, die das gemeinsame Erstellen von Prototypen stark verbessern. Das Tool ist von Designern für Designer entwickelt wurden und einfach im handling. Benutzeroberflächen lassen sich schnell und effektiv designen, ohne viel Zeit in die Einarbeitung des Tools investieren zu müssen. Das kollaboratives Arbeiten in Echtzeit ermöglicht die unkomplizierte Zusammenarbeit am selben Entwurf, ohne extra ein Tool zu installieren oder sich physisch treffen zu müssen. Des weiteren bietet Figma die Möglichkeit, erstellte Screens in vier verschiedenen Formaten zu exportieren als auch Interaktionen hinzuzufügen, um das Projekt im Live-Preview betrachten zu können. Im Live-Preview lassen sich alle zuvor definierten Interaktionen, wie beispielsweise das Navigieren auf einen anderen Screen, ausprobieren und imitieren das Feeling einer echten Anwendung.

\subsection{Mobile First}
Da unsere Zielgruppe eine breite Masse an Menschen beinhaltet, haben wir wert darauf gelegt, dass Xpire möglichst einfach und unkompliziert verwendet werden kann. Einem ansprechenden User Interface sowie einer intuitive User Experience werden daher besondere Bedeutung zugeschrieben.\\
Im Jahr 2015 meldete Google erstmals, dass mehr Suchanfragen über mobile Endgeräte als über Desktop-Geräte erfolgten. Seitdem steigt die der Anteil der Internetnutzer, die auch mit dem Smartphone online gehen, rapide. \footnote{\url{https://de.statista.com/statistik/daten/studie/633698/umfrage/anteil-der-mobilen-internetnutzer-in-deutschland/}} Resultierende aus dieser Entwicklung ist die Wahl des Mobile-First-Ansatzes selbsterklärend. Inzwischen ist nicht mobil-optimierter (responsive) Content ist kaum noch vorstellbar. Mobile Frist lässt sich als neuer Denkansatz im Webdesign definieren. Das Design einer Website wird dabei erstmal in der mobilen Version optimiert, bevor es für größere Bildschirme entwickelt wird. Man arbeitet also von der kleinsten Layout-Version hin zur größten.\autocite[vgl.][]{?}

% Schrittfolge der Screens erklären
\begin{figure}[hbt!]
	\centering
	\includegraphics[width=1.0\textwidth]{img/Prototype_01.pdf}
	\caption{Xpire Welcome-Screen und Home-Screen}
	\label{fig:prot1}
\end{figure}
Die Abbildung \ref{fig:prot1} zeigt den \textit{Welcome-Screen} und den \textit{Home-Screen} der Xpire-App. Der \textit{Welcome-Screen} zeigt das Xpire-Logo, welches mit dem vektorbasierten Grafik- und Zeichenprogramm \enquote{Adobe Illustrator} erstellt wurde. Daher kann es in unterschiedlichen Formaten exportiert und schnell angepasst oder verändert werden. Die gewählte Farbkombination als auch die verwendete Symbolik verleihen dem Logo einen hohen Wiedererkennungswert und ein gewisses Öko-Flair. Das Motto \enquote{Manage your fridge \& get rich} ist nicht nur ein einprägsamer Reim, der sich gut als Werbespruch verwenden lässt, gleichzeitig stellt er eine Anspielung der monetären Ersparnisse dar, die man durch die Benutzung der Xpire-App erzielen kann.\\
Der \textit{Home-Screen} in Abbildung \ref{fig:prot1} zeigt die aktuell hinterlegten Produkte mit Bild, Produktnamen, Verfallsdatum und einer farbigen Statusbar. Ist die Statusbar grün, bedeutet das, dass das Produkt noch haltbar ist und auch nicht unmittelbar vor dem Ablauf der Haltbarkeit steht. Ist der Balken gelb, wie im Mockup bei der \enquote{Milbona Schlagsahne} bedeutet dies, dass das Produkt kurz vor dem Ablauf der Haltbarkeit steht. Wechselt ein Produkt in den gelben Status, erhält der Benutzer eine Push-Notification, mit dem entsprechenden Hinweis zum Haltbarkeitsablauf des jeweiligen Produktes. Ist der Balken rot gefärbt, hat das Produkt das entsprechende Verfallsdatum bereits erreicht oder sogar überschritten.\\
Der \textit{Delete-Screen} zeigt, wie man ein Produkt durch wischen nach links löschen kann. Da sich die Anforderungen an die Bedienbarkeit in der Desktop-Version von der Smartphone-Version unterscheiden, gibt es zum Löschen eines Produktes in der Desktop-Version ein Mülleimer-Icon.
Wird im \textit{Home-Screen} ein Produkt angeklickt, so wird der Nutzer zum \textit{Product-Screen} des jeweiligen Produktes geleitet. Diese Ansicht enthält Details zum Produkt, wie Name, Anzahl, Einkaufdatum und Verfallsdatum und erlaubt dem Nutzer ein Bild des Produktes zu hinterlegen. Alle hinterlegten Informationen lassen sich in dieser Ansicht vom Nutzer ändern.


\begin{figure} 
	\centering
	\includegraphics[width=1.0\textwidth]{img/Prototpye_02.pdf}
	\caption{Xpire Delete-Screen und Product-Screen}
	\label{fig:prot2}
\end{figure}
