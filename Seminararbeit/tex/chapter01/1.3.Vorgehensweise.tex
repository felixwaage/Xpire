% !TEX root =  master.tex
\section{Vorgehensweise}
Um so effektiv wie möglich zusammen zu arbeiten, haben wir uns eine flexible Art der Aufgabenaufteilung überlegt. Dabei sind wir wie folgt vorgegangen: Zunächst haben wir gemeinsam in einer Brainstorming-Session überlegt, welche Funktionalitäten die Xpire-App beinhalten soll, wie sie aussehen soll und welche Technologien wir für die Umsetzung verwenden möchten. Anschließend hat sich jedes Gruppenmitglied selbstständig mit den Grundlagen einer PWA als auch mit der Softwarebibliothek React vertraut gemacht. In einer zweiten Session haben wir gemeinsam die Code-Basis für das Projekt aufgesetzt und spezifische Verantwortungsbereiche im Projekt definiert, zu denen sich jeder, je nach Stärken und Interessen, selbst zuordnen konnte. Dabei ist zu beachten, dass eine Person nicht zwangsläufig nur im eigenen Verantwortungsbereich agieren muss. Jeder darf entsprechend seiner Stärken den Fokus setzen und darüber hinaus weiteren Aufgaben außerhalb seines Schwerpunktes nachgehen und die anderen Team-Mitglieder bei ihren Aufgaben unterstützen. Wichtig ist uns, dass sich jeder im Team wohl fühlt, dem nachgehen kann, worauf er Lust hat und, ganz wichtig, durch Learning-by-Doing ganz viel Neues erlernen kann.\\
Die Schwerpunkte wurden wie folgt eingeteilt:
\begin{itemize}[noitemsep]
	\item \textbf{Design:} Verena, Andrea
	\item \textbf{App-Entwicklung:} Felix, Milena, Yvonne, Verena
	\item \textbf{Datenbank:} Fabio, Jonas
	\item \textbf{Projektmanagement:} Andrea
\end{itemize}
Freitags, an unseren offiziellen Meeting-Terminen, stellen wir einander vor, welche Aufgaben in der Zwischenzeit umgesetzt wurden, definieren neue Aufgaben und reflektieren die bisherige Vorgehensweise. Unter der Woche vereinbaren wir je nach Bedarf individuell Termine mit dem gesamten Team oder treffen uns in Teilgruppen, um bestimmte Aufgaben umzusetzen. Ein Protokoll der Meetings liegt in unserem GitHub-Repository im Dokument \enquote{meetings.md} und kann unter folgendem Link gefunden werden: https://github.com/felixwaage/Xpire/blob/master/meetings.md

Das gesamte Projekt kennzeichnet sich durch eine iterative Arbeitsweise, sodass genug Platz für neue Ideen und Erweiterungen bleibt. Ziel ist es, zunächst ein Minimal-Viable-Product (MVP) zu entwickeln, um ein funktionierendes Produkt liefern zu können, welches sich in zukünftigen Schritten einfach erweitern lässt.


