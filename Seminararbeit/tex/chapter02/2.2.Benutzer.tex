% !TEX root =  master.tex
\section{Benutzerprofilanalyse}
Im Rahmen der Benutzerprofilanalyse werden die Eigenschaften der Endanwender untersucht, um die Gebrauchstauglichkeit der Anwendung sicherzustellen. Eine wesentliche
Rolle spielen dabei die persönlichen Merkmale, Fähigkeiten und Kenntnisse der Benutzer
sowie deren Einordnung in ein bestimmtes Nutzungsverhalten.\\
Da die progressive Web App von möglichst vielen Nutzern verwendet werden soll, gibt es kaum Einschränkungen, was die Zielgruppe betrifft. Die Gemeinsamkeit der Nutzer liegt im Besitz eines Kühlschrankes, dem selbstständigem Tätigen von Lebensmittel-Einkäufen sowie im Besitz eines Smartphones oder eines Computers. Daraus lässt sich ableiten, dass die Zielgruppe überwiegend volljährig ist und ein Smartphone oder einen Computer besitzt. Laut Statista besitzen aktuell 66,5 Mio Menschen in Deutschland ein Smartphone.\footnote{\url{https://de.statista.com/statistik/daten/studie/500579/umfrage/prognose-zur-anzahl-der-smartphonenutzer-in-deutschland/}} Ca. 13 Mio Menschen in Deutschland sind Minderjährig, wobei nicht alle Minderjährigen ein Smartphone besitzen. Zieht man trotzdem die Anzahl aller Minderjährigen von der Anzahl der Menschen, die in Deutschland ein Smartphone besitzen, ab, kommt man immer noch auf eine Anzahl von mindestens 49 Mio potentiellen Nutzer. Das sind  mehr als 50\% aller deutschen Staatsbürger.\\
All diese Nutzer können einen völlig unterschiedlichen Erfahrungsschatz bezüglich der Nutzung von neuen Technologien aufweisen. Von der Rentnerin, die nur selten einen Computer verwendet bis hin zum zwanzigjährigen Studierenden, der selbstständig Web Apps programmiert, können die unterschiedlichsten Ausgangsvoraussetzungen gegeben sein. Es is davon auszugehen, dass ein Großteil der Anwendet eine durchschnittliche Technikaffinität aufweist, es aber auch Anwender gibt, die keinerlei oder kaum Erfahrungen im Umgang mit Web Apps aufweisen. Demzufolge sollte die Oberfläche gleichzeitig so konzipiert sein, dass sie auch für Laien intuitiv und verständlich ist. Ein strukturierter und klarer Aufbau könnten diesem Aspekt Rechnung tragen. Schließlich sei noch zu erwähnen, dass die Web App vorerst nur für Deutschland entwickelt wird. Zu einem späteren Zeitpunkt ist die Internationalisierung der Applikation durchaus denkbar, dafür bedarf es dann einer Möglichkeit zur Auswahl unterschiedlicher Sprachen.



